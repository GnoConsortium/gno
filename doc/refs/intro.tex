%
% GNO Overview
%
% $Id: intro.tex,v 1.1 1997/11/24 05:07:27 gdr Exp $
% 

\documentstyle{report}
\begin{document}

\title{GNO/ME Overview Version 2.0}
\author{Jawaid Bazyar \\ Tim Meekins}
\date{August 1996}
\maketitle

\parindent=0pt
\parskip=1pc

The GNO Multitasking Environment is Copyright 1991-1997 by
Procyon Enterprises Incorporated.

Documentation, second edition, August 1996.

The ORCA/C run-time libraries are Copyright 1987-1997 Byte Works, Inc.,
and distributed with permission.

This product includes software developed by the University of California,
Berkeley and its contributors.

GNO/ME 2.0 also includes several utilities and libraries produced by
outside authors and in the public domain. This software is included
solely as a convenience to users of GNO/ME, and is not considered part
of GNO/ME for copyright purposes.

GNO and GNO/ME are trademarks of Procyon Enterprises Incorporated.

Apple IIGS, APW, Finder, GS/OS, ProDOS, Macintosh, and HFS are registered
trademarks of Apple Computer, Inc.

AppleWorks GS is a trademark of Claris Corp.

UNIX is a registered trademark of AT\&T Bell Laboratories.

\bf
Important Notice:
\rm
This is a fully 
copyrighted work and as such is protected under the copyright
laws of the United States of America. According to these laws,
consumers of copywritten material may make copies for their
personal use only. Duplication for any other purpose whatsoever
would constitute infringement of copyright laws and the offender
would be liable to civil damages of up to \$50,000 in addition to
actual damages, plus criminal penalties of up to one year
imprisonment and/or a \$10,000 fine.


Procyon Enterprises Inc.
MAKES NO
WARRANTIES, EITHER EXPRESS OR IMPLIED, REGARDING THE ENCLOSED
COMPUTER SOFTWARE PACKAGE, ITS MERCHANTABILITY OR ITS FITNESS FOR
ANY PARTICULAR PURPOSE. THE EXCLUSION OF IMPLIED WARRANTIES IS
NOT PERMITTED IN SOME STATES. THE ABOVE EXCLUSION MAY NOT APPLY
TO YOU. THIS WARRANTY PROVIDES YOU WITH SPECIFIC LEGAL RIGHTS.
THERE MAY BE OTHER RIGHTS THAT YOU MAY HAVE WHICH VARY FROM STATE
TO STATE.


This product is sold for use on a
\em single computer
\rm at a 
\em single location .
\rm For information on obtaining a site license for using multiple copies,
contact the publisher.

% \begin{verbatim}
\leftline{Procyon Enterprises, Inc.}
\leftline{P.O. Box 641}
\leftline{Englewood, CO 80151-0641 USA}
\leftline{(303) 781-3273}
% \end{verbatim}
\vfill

\chapter{Credits}

\begin{quote}
Always listen to experts. They'll tell you what can't be done, and why.
Then do it.

\em Lazarus Long \rm
\end{quote}

The {\bf GNO} {\bf M}ultitasking {\bf E}nvironment for the Apple IIgs.

Copyright 1991-1997, Procyon Enterprises Inc. and Tim Meekins.

Please direct all inquiries to:
\tt \begin{verbatim}
	Procyon, Inc.
        PO Box 641
        Englewood, CO 80151-0641 USA
        (303) 781-3273
\end{verbatim} \rm 

For on-line technical assistance, contact:
\begin{tabbing}
America OnLine \=GNOJawaid, GNOTim2 \\
GEnie \>Procyon.Inc \\
Internet \>bazyar@hypermall.com \\
Delphi \>JAWAIDB \\
\end{tabbing}



\chapter{Miscellaneous}

\section{Reporting Bugs}

In any large piece of computer software
such as the GNO/ME system, bugs are sure to turn up, no matter
how much testing is performed on the software before it goes out
the door. If you discover a bug in GNO/ME, we'd like to hear
about it. There are several things we require, however, to make
bug reports useful to us.

First of all, we need a complete
description of your computer system; how much RAM, what cards, in
what slots, what type of disk storage, etc. This information is
very important in tracking down hardware-dependent bugs. Also, we
need to know the version numbers of the software involved; the
GNO Kernel, the GNO Shell, and any utilities.

Second, we need a step-by-step description
of how to duplicate the bug. If this requires writing down
individual keystrokes, then we need it. Only in this way can we
decide whether the bug is hardware dependent or global.

You can send bug reports to any of the
electronic mail addresses listed on the 'Credits' page, or by
mailing a disk containing a description of the problem and the
necessary software and files to the Procyon address.

\section{User Projects}

If you're working on a project which
utilizes GNO/ME in some way, we'd like to know about it. Just
contact us by mail, phone, or whatever is most convenient for
you, and tell us about your project. If possible, we'll
coordinate your efforts with those of other programmers. If we
really like your project and think it may be useful to others, we
may include it on the next GNO/ME distribution!

\section{Software Piracy}








If you have illegally copied computer software from someone
and are now reading this, take a moment to reflect on what you've done.
Many computer software applications are huge projects, consuming
many man-years of effort, huge amounts of money, and a lot of grief on the
part of the developers.
For example, the Byte Works' ORCA languages and development 
environment (as of November 1997) consists of over 220,000 lines
of source code, \em not \rm including the libraries, tool 
interface files, test programs, samples, or the 5 800k disks
of source in the courses for each language.  This work represents
approximately 17 man-years of development, support, marketing,
and management.

Is it right that you're now benefiting from
developers' efforts without any just compensation to the authors?

Many programmers are born, bred, and raised in the spirit of
computing. They love to write software, and probably always
will. But if they cannot make enough money to make it
worthwhile to continue their work, then they won't; because by
worthwhile, we mean not only food on the table, but resources for
expansion and continued growth. In short,
support them and they'll support you.
 
Computer software is very inexpensive when you
consider what it allows you to do.
 
To all you who properly pay for the software you use:
Thank you.

\chapter{Preface}

Computers are tools. The flexibility of a
tool determines how useful it is. Early computers were much like
the one this software was written for: the Apple IIgs. They could
only run one program at a time, and their usefulness was limited
to what the particular program the user was executing offered. In
the late 1960's, a team of researchers at AT\&T began
developing the UNIX operating system. The UNIX design was
partially based on the premise that most programs are I/O bound,
that is, most of the time the program executes is spent waiting
for user input or other I/O operations. While one program is
waiting for I/O, why not allow another program to execute? This
is what they did, and the result was one of the most successful
computer operating systems ever created.

The Apple IIgs, like the Macintosh it is
modelled after, provides very limited multitasking abilities in
the form of desk accessories (NDAs). The programs in the NDA menu
are available in whatever application you use as long as it
follows Apple's guidelines. However, there are many graphics
based programs that don't support NDAs, and in addition there is
a wealth of software that has been developed for the Byte Works'
ORCA environment. This environment is mainly text-based, and thus
makes access to NDAs impossible. As if that wasn't enough, it's
very difficult to write an NDA to allow the application to keep
running concurrently. So the benefits are lost, and we're back at
ground zero.

Enter the GNO Multitasking Environment.
What was once just dreamed about is now a reality. GNO/ME
provides an environment that is almost entirely compatible with
software developed for the ORCA environment. But GNO/ME also
provides a wealth of new abilities, lots of new ground for
developers and users alike.

Before we begin describing, we'd like to
respond to those who say such a multitasking system isn't
possible on the Apple IIgs. Obviously it is: you hold it in your
hands. Some say the Apple IIgs isn't powerful enough to make
multitasking useful. We point out that the Apple IIgs is much
more powerful than the first computers UNIX was designed to run
on; they only had 64K of real memory, and were 16 bit machines.
Some ask why you'd ever need to run more than one program at
once. These are the same people who asked why we'd ever need more
than 64K of memory, or more than 140K of storage on disks (end
soapbox).

\chapter{Introduction}

The GNO Multitasking Environment provides
pre-emptive multitasking. Many programs can be executing at the
same time; each is called a 'process'. Each process is allowed to
run for a short period of time (1/20th of a second on average).
When its time runs out, the current process is set aside and
another one chosen to run next. This cycle continues until there
are no more processes left (i.e. when you exit GNO/ME). Starting
up processes to run 'in the background' is a simple matter of
adding a few characters to the shell commands.

GNO/ME provides a shell that takes full
advantage of the multitasking ability provided. The most
important feature of the shell is job control (starting,
terminating, and suspending processes). But the shell also
provides power never before seen on the Apple IIgs. The ability
to choose files by 'wildcard' has been around for a while, but
the GNO Shell takes this to a new level with 'regular
expressions', a very powerful yet simple programming language.
Other benefits of the GNO shell are too numerous to mention. (see
the \bf GNO Shell User's Manual \rm for details).

In addition to being compatible with the
ORCA system, GNO/ME is a very powerful programming environment.
Available to the programmer are all the calls needed to control
processes, support Inter-Process Communication, and other tools
needed in a multitasking environment.

GNO/ME also boasts the first completely
consistent method for accessing serial and console I/O. The IIgs
TextTools have been incredibly enhanced to provide a truly
all-encompassing interface for serial, console, and IPC
applications. Imagine being able to attach terminals to your GS,
and have a useful shell in each one. Multiuser BBSs, remote
dial-ups, UUCP or SLIP that doesn't take over your computer- the
applications are endless!

With all this talk of shell utilities, have
desktop users (users of programs like AppleWorks GS) been left
behind? Absolutely not. GNO/ME doesn't allow more than one
desktop program to run concurrently, but it DOES let you run a
desktop program with as many text applications as you like. In
other words, no functionality is lost from the IIgs by using
GNO/ME.

Finally, the GNO Multitasking Environment
comes with a large number of free utilities that bring some of
the power of a UNIX system to the Apple IIgs. Also, a number of
programming libraries are included that make it easy to port
programs from UNIX or MS-DOS systems to the Apple IIgs.

\chapter{The GNO/ME Package}

Included in your GNO/ME Version 2.0 package are:
\begin{itemize}
\item	This GNO/ME overview.
\item	The GNO Shell User's Manual.
\item	The GNO Kernel Reference Manual.
\item	A selection of utility and library documentation.
\item	A reading list containing a wide selection of books 
	for both the user and the programmer
\item	Three disks containing the GNO Kernel, GNO Shell, and 
	loads of utilities.
\end{itemize}

\chapter{Hardware Requirements}

GNO/ME will work on any Apple IIgs with at least 2 MegaBytes of
memory and a hard drive.

You should have at least 5 MegaBytes of hard disk space free.

We recommend 4 MegaBytes of Memory and an 
accellerator card, especially if you will be using GNO with many
background processes. A modem and access to an on-line service
will greatly speed access to technical assistance and new
utilities as they are made available.


\parindent=20pt

\end{document}
